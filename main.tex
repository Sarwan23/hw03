%CS-113 S18 HW-3
%Released: 16-Feb-2018
%Deadline: 2-March-2018 7.00 pm
%Authors: Abdullah Zafar, Waqar Saleem.


\documentclass[addpoints]{exam}

% Header and footer.
\pagestyle{headandfoot}
\runningheadrule
\runningfootrule
\runningheader{CS 113 Discrete Mathematics}{Homework II}{Spring 2018}
\runningfooter{}{Page \thepage\ of \numpages}{}
\firstpageheader{}{}{}

\boxedpoints
\printanswers
\usepackage[table]{xcolor}
\usepackage{amsfonts,graphicx,amsmath,hyperref}

\title{Habib University\\CS-113 Discrete Mathematics\\Spring 2018\\HW 3}
\author{$ss03595$}  % replace with your ID, e.g. oy02945
\date{Due: 19h, 2nd March, 2018}


\begin{document}
\maketitle

\begin{questions}



\question
All sets carry data, but how much information can be extracted from it? Consider a simple model on a set $A$, in which each relation encodes 1 unit of information. We define the ``Information Potential" of a set as the sum of information units (or the number of distinct relations) that can be generated from the set. In the questions that follow, you may assume all relations to be binary.

\begin{parts}
  \part Consider $A$ to be the set of $n$ distinct facts. What is the information potential of this set?
  
  \begin{solution}
    % Write your solution here
    Since $R \subseteq AxA$, where R is relation on set A. We know that AxA has $n^2$ elements/information units, and a set with m elements/information units has $2^m$ subsets. Hence, the set A has a information potential of $2^{n^2}$.
  \end{solution}
  
  \part Reflexive pairs of the form $(fact\;x, fact\;x)$ are considered redundant in our model. What is the information potential of the ``non-redundant" set, that is, the set without reflexive relations? 
  
  \begin{solution}
    % Write your solution here
    Considering our set to be A, with n elements. There will n reflexive pairs in the set AxA, since each element forms a reflexive pair with itself. As such, the information potential of the 'non-redundant' set should be $2^{n^2 - n}$
  \end{solution}

  \part Anti-symmetric relations that follow the rule $(fact\;x,fact\;y)\; \land (fact\;y,fact\;x) \rightarrow fact\;x = fact\;y$ are of special interest to our model. Such pairs, as in the aforementioned antecedent, can be used to express ordered relationships between facts. What is the combined Information Potential of anti-symmetric relations on the non-redundant set? 
  \begin{solution}
    % Write your solution here
    \\Considering R is an anti-symmetric relation on A
    \\Suppose $A = \{\}$ then $|R| = 0$
    \\if $A = \{a\}$ then $|R| = 0$
    \\if $A = \{a, b\}$ then $|R| = 1$
    \\if $A = \{a, b, c\}$ then $|R| = 3$
    \\and so on the pattern follows as 6, 10, 15, 21.
    \\From this, we can deduce, $|R| = n(n-1)/2$
    \\Taking any two different elements $x, y \in A$. Antisymmetry demands that one of the following must be true:
    \\(1) Neither (x, y) nor (y, x) belongs to R.
    \\(2) (x, y) belongs to R, but (y, x) does not.
    \\(3) (y, x) belongs to R, but (x, y) does not.
    \\Hence, combined Information Potential = $3^{(n(n-1)/2)$

  \end{solution}
  
  \part There are many ways to describe a relation in natural language. For example, a relation described as $``x<y"$ over the set $\{1,2\}$ may also be described as $``x+1=y"$. Specifically, two descriptions that produce the same relation are considered ``isomorphs" of one another in our model. There may be any number of isomorphs for a given relation. Given $2^{n^2+1}$ descriptions of relations, how many isomorphs exist? (Give your answer as a range)
  
  \begin{solution}
    % Write your solution here
    We know that the total numbers of relations are $2^{n^2}$, and the total number of description's is $2^{n+1}$.
    This can also be written as $2^{1}$ x $2^{n^2} = 2^{n^2} + 2^{n^2}$. If there are two descriptions of the same relation then it has 2 isomorphs, and if one, then 0 isomorphs. Hence, the maximum number of isomorphs is the number of descriptions itself since all of them may be describing only a single relation: $2^{n+1}$
    \\
    \\The minimum number of isomorphs would be: $2^{n} + 1$ because only one of the relations 
    \\
    \\ $2^{n} + 1 <= isomorphs <= 2^{n+1}$
  \end{solution}

\end{parts}

\question Let $R$ be a relation from $A$ to $B$. Then the inverse of $R$, written $R^{-1}$, is a relation from $B$ to $A$ defined by $R^{-1} = \{(y,x) \in B \times A \:|\: (x,y) \in R\}$. Prove that $R$ is symmetric iff $R = R^{-1}$.

  \begin{solution}
    % Write your solution here
    \\To prove that R is symmetric iff $R = R^{-1}$, we must first prove that $(R \subseteq R^{-1}) \wedge (R^{-1} \subseteq R)$, that is if R is symmetric then $R = R^{-1}$. 
    \\
    \\Suppose $(a, b) \in R $ where $a \in A$ and $b \in B$
    \\then, $(b, a) \in R$ assuming R is symmetric
    \\and, $(a, b) \in R^{-1}$, by definition of inverse: $b R^{-1} a$ iff $a R b$
    \\Hence, $R \subseteq R^{-1}$
    \\
    \\Now suppose $(a,b) \in R^{-1}$, where $a \in B$ and $b \in A$.
    \\then $(a, b) \in R$, by definition of inverse
    \\and, $(b, a) \in R$ because R is symmetric.
    \\Hence,$R^{-1} \subseteq R$
    \\
    \\Since both are subsets of each other, then $R = R^{-1}$
    \\Next, we prove that if $R = R^{-1}$, then R is symmetric.
    \\
    \\Suppose $(a, b) \in R $ where $a \in A$ and $b \in B$
    \\then, $(b, a) \in R^{-1}$ by definition of inverse
    \\and, $(b, a) \in R$ assuming $R = R^{-1}$
    \\
    \\Since both $((a, b) \wedge (b, a)) \in R$, R is symmetric.
  \end{solution}

\question Let $R$ and $S$ be relations on a set $A$. Assuming $A$ has at least 3 elements, state whether each of the following statements is true or false, providing a brief explanation if true, or a counterexample if false:
\begin{parts}
\part If $R$ and $S$ are reflexive, then $R \cup S$ is reflexive.
\part If $R$ and $S$ are anti-symmetric, then $R \circ S$ is anti-symmetric.
\part If $R$ and $S$ are symmetric, then $R \cap S$ is symmetric.
\part If $R$ is reflexive, then $R \cap R^{-1}$ is not empty. 
\part If $R$ is transitive, then $R^{-1}$ is transitive.
\end{parts}


  \begin{solution}
    % Write your solution here
    \\a) True, because $\forall x \in A$ by definition of reflexivity $(x, x) \in R$ and $(x, x) \in S$. Hence, $(x, x) \in (R \cup S)$. i.e $(x, x) \subseteq R \subseteq (R \cup S)$. $R \cup S$ contains all the order pairs such that $(x, x)$ and hence, it is reflexive too.
    \\
    \\b)False, suppose $A = \{1, 2, 3\}$
    \\$R = \{(1, 2), (1, 3), (2, 3)\}$ 
    \\$S = \{(2, 1), (3, 1), (3, 2)\}$
    \\Both are anti-symmetric.
    \\$R \circ S = \{(1, 1), (1, 1), (1, 2), (2, 2), (2, 1)\}$
    \\Since, $(1, 2) \in (R \circ S)$ and $(2, 1) \in (R \circ S)$, $R \circ S$ is not anti-symmetric.
    \\
    \\c)True, because for an arbitrary $(a, b) \in A$ by definition of symmetry $(a, b) \in R$ and $(b, a) \in R$, similarly if $(b, a) \in S$, then $(a,b) \in S$. Since, $((a, b) \wedge (b, a)) \in R \cap S$ it then follows that $R \cap S$ is symmetric.
    \\
    \\
    \\d)True, because $\forall x \in A$ by definition of reflexivity $(x, x) \in R$ and by definition of inverse: $b R^{-1} a$ iff $a R b$, as in our case $R^{-1}$ contains all ordered-pairs $(a, b)$ such that $a = b$ since $(a = x \wedge b = x)$ and hence,  $\forall x \in A$ by definition of reflexivity $(x, x) \in R^{-1}$ Conclusively, $R \cap R^{-1}$ cannot be empty as R and $R^{-1}$ share common pairs.
    \\
    \\e)True, because by definition of transitivity $(x,y),(y,z) \in R \implies (x, z) \in R$. Then, by definition of inverse $(z,y),(y,x) \in R^{-1} \implies (z, x) \in R^{-1}$. Thus, it is transitive too.
  \end{solution}
  
\question Let $R$ be a relation on $A$. Prove that the digraph representation of $R$ has a path of length $n$ from $a$ to $b$ iff $(a, b) \in R^n$.
 
  \begin{solution}
    % Write your solution here
    \\Inducive Hypothesis:
    \\Suppose that the theorem is true for the positive integer n.
    \\
    \\ There is a path of length n + 1 from a to b if and only if there is an element $x \in A$ such that
    there is a path of length 1 from a to x, so $(a, x) \in R$, and a path of length n from x to b,
    that is, $(x, b) ∈ R^n$. 
    \\
    \\As such by the inductive hypothesis, there is a path of length n + 1 from a to b if and only if there is an element c with $(a, x) \in R$ and $(x, b) \in R^n$. But there
    is such an element if and only if $(a, b) \in R^{n+1}$. Therefore, there is a path of length n + 1
    from a to b if and only if $(a, b) \in R^{n+1}$. Hence proved.
    
  \end{solution}

\question
    Let $R$ be a relation on a set $A$. We define

    $\rho (R) = R \cup \{(a, a) | a \in A\}$ \\ 
    $\phi (R) = R \cup R^{-1}$ \\
    $\tau (R) = \cup \{ R^n | n = 1,2,3,...\}$
    
    Show that $\tau (\phi (\rho (R)))$ is an equivalence relation containing $R$.
    
      \begin{solution}
    % Write your solution here
    By definition of union, we know that $\forall a \in A ((a, a) \in \rho (R))$ where $a = a$. Hence, we see that $\rho (R)$ is reflexive.
    \\
    \\We know that for an arbitrary $(a, b) \in (\phi (\rho (R)))$, by definition of union and inverse of a relation: $b R^{-1} a$ iff $a R b$, that $(b, a) \in (\phi (\rho (R)))$. This shows it is symmetric
    \\Furthermore, it follows that in $\phi (R)$, $R = \rho (R) $,  which contains $(a, a)$. And since $\phi (R)$ is defined as $R \cup R^{-1}$, then it follows that $(\phi (\rho (R)) = \rho (R) \cup \rho (R)^{-1}$. Conclusively, by definition of union $(a, a) \in (\phi (\rho (R))$ and as such $(\phi (\rho (R))$ is maintains reflexivity too.
    \\
    \\Theorm: The relation R on a set A is transitive if and only if $R^n \subseteq  R$ for n = 1, 2, 3,....
    \\$\tau (\phi (\rho (R)))$ is the union of  $R^1 \cup R^2 \cup R^3 ... R^n$ where $R = \phi (\rho (R))$ which we know is reflexive and symmetric. If we can prove that any $R^{n} \subseteqof R$, then it proves that $\tau (\phi (\rho (R)))$ is transitive.
    \\
    \\1) $(R^{n} \subseteq R)\implies \tau (\phi (\rho (R)))$ is transitive
    \\Suppose $(a, b) \in R$ and $(b, c) \in R$, then by definition of composition
    \\$(a,c) \in R^2$ and since we are assuming $R^2 \subseteq R$, then $(a,c) \in R$. \\Hence, R is transitive and so is $\cup \{ R^n | n = 1,2,3,...\}$
    \\
    \\2)Now we need to prove $\tau (\phi (\rho (R)))$ is transitive $\implies (R^{n} \subseteq R)$
    \\Suppose $(a, b) \in R^{n+1}$, because $R^{n+1} = R^n \circ R$, then there must be a $(a, x) \in R$ and $(x, b) \in R^n$ such that by definition of composition $(a, b) \in R^{n+1}$. Since R is transitive, then $(a, x) \in R$ as well as $(b, x) \in R$, and as such $(a, b) \in R$. Hence, $R^{n+1} \subseteq R$
    \\
    \\To conclude, since $\tau (\phi (\rho (R)))$ is all three: reflexive, transitive and symmetric, it is an equivalence relation.
    
    
    
  \end{solution}


\end{questions}

\end{document}
